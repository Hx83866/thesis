
The prediction of the future course of a process instance has been an important subfield in process mining research, aiming to enhance process monitoring capabilities.
Depending on the use case, for example predicting time-related attributes, the future path or the the outcome of a case can be of interest.
Most approaches presented in the literature either use machine learning or process models to construct a predictor, that generalizes from a historical event log.

Five different non-parametric regression predictors for forecasting the total cycle time of an unfinished case have been presented by \Citeauthor{DBLP:conf/otm/DongenCA08} \cite{DBLP:conf/otm/DongenCA08}.
The estimates are based on activity occurrences, activity duration and attributes.

Furthermore, \Citeauthor{DBLP:journals/is/AalstSS11} proposed to build a transition system using a set, bag or sequence abstraction, which is annotated with time-related data in order to predict the cycle time of case \cite{DBLP:journals/is/AalstSS11}.
The core idea of this approach is to replay unfinished cases on the learned transition system and compute the prediction using the annotated data.

\citeauthor{DBLP:conf/colcom/PandeyNC11} use a hidden markov model to predict the cycle time of a case using the activity and timestamp data of an event log \cite{DBLP:conf/colcom/PandeyNC11}.

\citeauthor{DBLP:conf/icsoc/Rogge-SoltiW13} showed how a stochastic Petri net can be used to predict the cycle time of a process instance.
The model naturally supports parallelism in business processes and considers future events, which are expected to occur. 

\citeauthor{DBLP:conf/dis/CeciLFCM14} presented an approach, where a sequence tree is learned in order relate running traces to similar historical traces \cite{DBLP:conf/dis/CeciLFCM14}.
A decision tree is then used to predict the next activity and the cycle time of a case.

\citeauthor{DBLP:conf/bpm/TeinemaaDMF16} applied text vectorization techniques like bag-of-n-grams (BoNG), Latent Dirichlet Allocation (LDA) and Paragraph Vectors (PV) to textual data of processes in order to predict a binary label describing the process outcome \cite{DBLP:conf/bpm/TeinemaaDMF16}.
In this approach random forest and logistic regression classifiers for each prefix length of a trace are trained.

Most recently, several authors have applied recurrent neural networks in form of LSTM networks for process prediction. \citeauthor{ DBLP:conf/bpm/EvermannRF16} encode events using an embedding matrix as it is known for word embeddings. The embedded events are then used as input for an LSTM network that predicts the next activity \cite{DBLP:conf/bpm/EvermannRF16}.

\citeauthor{DBLP:conf/caise/TaxVRD17} use an one-hot-encoding of the activity and the timestamp of an event to predict the activity and timestamp of the next event.
This is done by using a two-layered LSTM network \cite{DBLP:conf/caise/TaxVRD17}.

The work by \citeauthor{DBLP:conf/ssci/NavarinVPS17} adopts the idea of using an LSTM network \cite{DBLP:conf/caise/TaxVRD17} and extends the encoding to additional data attributes associated with each event \cite{DBLP:conf/ssci/NavarinVPS17} to predict the cycle time of an case.

\citeauthor{DBLP:journals/computing/PolatoSBL18} presented a set of approaches that use support vector regression for cycle time prediction  \cite{DBLP:journals/computing/PolatoSBL18}.
In this work the authors implement different encoding for events including simple one-hot-encoding and a more advanced state based encoding using transition systems.
Furthermore, they enhance the approach in \cite{DBLP:journals/is/AalstSS11} by taking additional data attributes into account.

\citeauthor{DBLP:journals/tkdd/TeinemaaDRM19} reported an in-depth review and benchmark of outcome-oriented predictive process monitoring approaches.
The study showed that aggregated encoding like counting frequencies of activities as most reliable encoding for outcome-prediction \cite{DBLP:journals/tkdd/TeinemaaDRM19}.

\citeauthor{DBLP:conf/icpm/ParkS19} showed how LSTM-based predictions can be used to solve a resource allocation problem, leading to direct recommendations for process improvement \cite{DBLP:conf/icpm/ParkS19}.

A comparison of the process prediction methods is presented in Table \ref{tab:preliminaries}.


\begin{table}[]
	\renewcommand{\arraystretch}{1.5}
	\begin{tabularx}{\textwidth}{
			>{\hsize=2.0\hsize}X
			>{\hsize=0.4\hsize}X
			>{\hsize=1.3\hsize}X
			>{\hsize=0.5\hsize}X
			>{\hsize=0.5\hsize}X
			>{\hsize=1.3\hsize}X
		}
		\toprule
		\textbf{Contribution} & \textbf{Year} & \textbf{Model(s)}  & \textbf{Data-Aware} &  \textbf{Text-Aware} & \textbf{Predictions} \\ \midrule
		 Van Dongen et al. \cite{DBLP:conf/otm/DongenCA08}& \citeyear{DBLP:conf/otm/DongenCA08} & Regression  & \checkmark & \xmark& Cycle time\\
		 
		 Van der Aalst et al. \cite{DBLP:journals/is/AalstSS11}&  \citeyear{DBLP:journals/is/AalstSS11}& Transition system  
		   & \xmark & \xmark & Cycle time \\   
		   
		 \citeauthor{DBLP:conf/colcom/PandeyNC11} \cite{DBLP:conf/colcom/PandeyNC11} & \citeyear{DBLP:conf/colcom/PandeyNC11} & Hidden Markov & \xmark & \xmark & Cycle time \\
		 
		 \citeauthor{DBLP:conf/icsoc/Rogge-SoltiW13} \cite{DBLP:conf/icsoc/Rogge-SoltiW13} & \citeyear{DBLP:conf/icsoc/Rogge-SoltiW13} &Stochastic Petri net & \xmark & \xmark & Cycle time\\
		 
		 \citeauthor{DBLP:conf/dis/CeciLFCM14} \cite{DBLP:conf/dis/CeciLFCM14} & \citeyear{DBLP:conf/dis/CeciLFCM14} & Sequence tree \newline Decision tree& \checkmark & \xmark & Next activity \newline Cycle time \\
		 
		 \citeauthor{DBLP:conf/bpm/TeinemaaDMF16}  \cite{DBLP:conf/bpm/TeinemaaDMF16} &  \citeyear{DBLP:conf/bpm/TeinemaaDMF16} & Random forest \newline Logistic regression & \checkmark & \checkmark & Case outcome \\
		 
		 \citeauthor{ DBLP:conf/bpm/EvermannRF16} \cite{ DBLP:conf/bpm/EvermannRF16} &  \citeyear{ DBLP:conf/bpm/EvermannRF16}& LSTM & \xmark & \xmark & Next activity \\
		 
		 \citeauthor{DBLP:conf/caise/TaxVRD17} \cite{DBLP:conf/caise/TaxVRD17} & \citeyear{DBLP:conf/caise/TaxVRD17} & LSTM & \xmark & \xmark & Next activity \newline Next event time \newline Cycle time \newline Future path \\
		 \citeauthor{DBLP:conf/ssci/NavarinVPS17} \cite{DBLP:conf/ssci/NavarinVPS17} &  \citeyear{DBLP:conf/ssci/NavarinVPS17}&  LSTM & \checkmark  & \xmark & Cycle time\\
		 
		 \citeauthor{DBLP:journals/computing/PolatoSBL18} \cite{DBLP:journals/computing/PolatoSBL18}&  \citeyear{DBLP:journals/computing/PolatoSBL18} &  Transition system \newline SVR \newline Naive Bayes &  \checkmark & \xmark & Next activity \newline Cycle time \newline Future path \\
		 
		 \citeauthor{DBLP:conf/icpm/ParkS19} \cite{DBLP:conf/icpm/ParkS19} & \citeyear{DBLP:conf/icpm/ParkS19} & LSTM & \checkmark & \xmark &Next activity \newline Next event time  \\
		 
		 This contribution &  2020 &  LSTM & \checkmark & \checkmark &  Next activity \newline Next event time  \newline Cycle time \newline Case outcome \newline Future path
	 	\\ \bottomrule
	\end{tabularx}
	\caption[Comparison  of process prediction methods]{Comparison  of process prediction methods.}
	\label{tab:preliminaries}
\end{table}