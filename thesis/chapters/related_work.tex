
The prediction of the future behavior of an process instance has been an important sub-field in the process mining research, that aims to enhance process monitoring capabilities.
Depending on the use case e.g. predicting time-related attributes, the future path or the the outcome of a case can be of interest.
Most approaches presented in the literature either use machine learning based or process models based methods.

\Citeauthor{DBLP:conf/otm/DongenCA08} presented five different non-parametric regression predictors for forecasting the total cycle time of an unfinished case\cite{DBLP:conf/otm/DongenCA08}.
The estimates are based on activity occurrences, activity duration and attributes.

\Citeauthor{DBLP:journals/is/AalstSS11} proposed to build a transition system using a set, bag or sequence abstraction, which is annotated with time-related data in order to predict the remaining time of case \cite{DBLP:journals/is/AalstSS11}.
The core idea of this approach is to replay unfinished cases on the learned transition system and compute the prediction using the annotated data.

\citeauthor{DBLP:conf/colcom/PandeyNC11} use a Hidden Markov Model to predict the remaining time of a case using the activity and timestamp data of an event log \cite{DBLP:conf/colcom/PandeyNC11}.

\citeauthor{DBLP:conf/icsoc/Rogge-SoltiW13} showed how a Stochastic Petri Net can be used to predict the remaining of a process instance.
The model naturally supports parallelism in business processes and considers future events a expected to occur. 

\citeauthor{DBLP:conf/dis/CeciLFCM14} presented an approach, where a sequence tree is learned in order relate a running traces to similar historical traces \cite{DBLP:conf/dis/CeciLFCM14}.
A decision tree is then used to predict the next activity and the remaining time of a case.

\citeauthor{DBLP:conf/bpm/TeinemaaDMF16} applied text vectorization techniques like bag of n-grams (BoNG), Latent Dirichlet Allocation (LDA) and Paragraph Vectors (PV) to textual data of processes in order to predict a binary label describing the process outcome\cite{DBLP:conf/bpm/TeinemaaDMF16}.
In this approach random forest and logistic regression classifiers for each prefix length of a trace are trained.

Most recently, several authors have applied recurrent neural networks in form of LSTM networks for process prediction. \citeauthor{ DBLP:conf/bpm/EvermannRF16} encode events using an embedding matrix as it is known for word embeddings. The embedded events are then used as input for an LSTM network that predicts the next activity\cite{DBLP:conf/bpm/EvermannRF16}.

\citeauthor{DBLP:conf/caise/TaxVRD17} use an one-hot-encoding of the activity and the timestamp of an event to predict the activity and timestamp of the next event.
This is done by using a two-layered LSTM network\cite{DBLP:conf/caise/TaxVRD17}.


The work by \citeauthor{DBLP:conf/ssci/NavarinVPS17} adopts the idea of using an LSTM network \cite{DBLP:conf/caise/TaxVRD17} and extends the encoding to additional data attributes associated with each event\cite{DBLP:conf/ssci/NavarinVPS17} to predict the remaining time of an case.

\citeauthor{DBLP:journals/computing/PolatoSBL18} presented a set of approaches that use support vector regression for remaining time prediction\cite{DBLP:journals/computing/PolatoSBL18}.
In this work the authors implement different encoding for events including simple one-hot-encoding and a more advanced state based encoding using transition systems.
Furthermore, they enhance the approach in  \cite{DBLP:journals/is/AalstSS11} by taking additional data attributes into account.

\citeauthor{DBLP:journals/tkdd/TeinemaaDRM19} reported an in-depth review and benchmark of outcome-oriented predictive process monitoring approaches.
The study showed that aggregated encoding like counting frequencies of activities as most reliable encoding for outcome-prediction \cite{DBLP:journals/tkdd/TeinemaaDRM19}.

\citeauthor{DBLP:conf/icpm/ParkS19} showed how LSTM-based predictions can be used to solve a resource allocation problem, leading to direct recommendation for process improvement \cite{DBLP:conf/icpm/ParkS19}.

A comparison of suggested process prediction methods is presented in Table \ref{tab:preliminaries}.

\begin{table}[]
	\renewcommand{\arraystretch}{1.5}
	\begin{tabularx}{\textwidth}{p{3.5cm} l p{3.2cm} p{1.1cm} p{1.1cm} p{3cm}}
		
		\toprule
		\textbf{Approach} & \textbf{Year} & \textbf{Model(s)}  & \textbf{Data-Aware} &  \textbf{Text-Aware} & \textbf{Predictions} \\ \midrule
		 \Citeauthor{DBLP:conf/otm/DongenCA08} \cite{DBLP:conf/otm/DongenCA08}& \citeyear{DBLP:conf/otm/DongenCA08} & Regression  & \checkmark & - & Remaining time\\
		 
		 \Citeauthor{DBLP:journals/is/AalstSS11} \cite{DBLP:journals/is/AalstSS11}&  \citeyear{DBLP:journals/is/AalstSS11}& Transition system  
		   & - & - & Remaining time \\   
		   
		 \citeauthor{DBLP:conf/colcom/PandeyNC11} \cite{DBLP:conf/colcom/PandeyNC11} & \citeyear{DBLP:conf/colcom/PandeyNC11} & Hidden Markov & - & - & Remaining time \\
		 
		 \citeauthor{DBLP:conf/icsoc/Rogge-SoltiW13} \cite{DBLP:conf/icsoc/Rogge-SoltiW13} & \citeyear{DBLP:conf/icsoc/Rogge-SoltiW13} &Stochastic Petri Net &- & -& Remaining time\\
		 
		 \citeauthor{DBLP:conf/dis/CeciLFCM14} \cite{DBLP:conf/dis/CeciLFCM14} & \citeyear{DBLP:conf/dis/CeciLFCM14} & Sequence Tree \newline Decision Tree& \checkmark & - & Next activity \newline Remaining time \\
		 
		 \citeauthor{DBLP:conf/bpm/TeinemaaDMF16}  \cite{DBLP:conf/bpm/TeinemaaDMF16} &  \citeyear{DBLP:conf/bpm/TeinemaaDMF16} & Random Forest \newline Logistic regression & \checkmark & \checkmark & Case outcome \\
		 
		 \citeauthor{ DBLP:conf/bpm/EvermannRF16} \cite{ DBLP:conf/bpm/EvermannRF16} &  \citeyear{ DBLP:conf/bpm/EvermannRF16}& LSTM & - & - & Next activity \\
		 
		 \citeauthor{DBLP:conf/caise/TaxVRD17} \cite{DBLP:conf/caise/TaxVRD17} & \citeyear{DBLP:conf/caise/TaxVRD17} & LSTM & - & - & Next activity \newline Future path \newline Next event time \newline Remaining time\\
		 \citeauthor{DBLP:conf/ssci/NavarinVPS17} \cite{DBLP:conf/ssci/NavarinVPS17} &  \citeyear{DBLP:conf/ssci/NavarinVPS17}&  LSTM & \checkmark  &- & Remaining time\\
		 
		 \citeauthor{DBLP:journals/computing/PolatoSBL18} \cite{DBLP:journals/computing/PolatoSBL18}&  \citeyear{DBLP:journals/computing/PolatoSBL18} &  Transition system \newline SVR \newline Naive Bayes &  \checkmark & - & Next activity  \newline Future path \newline Remaining time \\
		 
		 \citeauthor{DBLP:conf/icpm/ParkS19} \cite{DBLP:conf/icpm/ParkS19} & \citeyear{DBLP:conf/icpm/ParkS19} & LSTM & \checkmark & - &Next activity \newline Next event time  \\
		 
		 This approach &  2020 &  LSTM & \checkmark & \checkmark &  Next activity \newline Future path \newline Next event time  \newline Remaining time \newline Case outcome 
	 	\\ \bottomrule
	\end{tabularx}
	\caption{Comparison  of process prediction methods.}
	\label{tab:preliminaries}
\end{table}