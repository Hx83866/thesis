In this chapter, the performance of the text-aware process prediction model is evaluated based on real-life event data.
First, the data sets and evaluation method are described. Then, the performance of differently parameterized models on the data sets is evaluated and analyzed in-depth.


\section{Data Sets}

The prediction model is evaluated on real-world and simulated event logs, which are described in the following. An overview of the key properties of the logs is summarized in Table \ref{tab:logs}.

\textbf{Job Application} (Simulated) This event log describes a simple job application process. 
First, the applicant submits an application in the system.
Then, a curriculum vitae (CV) and a cover letter are uploaded in an random order, where the CV is uploaded first with 75\% probability.
When both documents have been received, the applicant is either directly rejected or invited to an interview.
After the interview a decision is made and the applicant gets an job offer or is rejected.
The CV contains text information about the skills of the applicant.
The probability for an interview invitation is exactly the relative share of technology related skills.
After the interview, an email is sent to the applicant with the final decision.
After that, the applicant gets an job offer or is rejected.
Furthermore, noise is added by introducing a 3\% probability that the applicant will stop the process after every event.
The time of each event is determined by sampling from a normal distribution, which is unique per activity.
The text information in the CV and the email is generated by sampling multiple times from sets of sentences.
For example, if an applicant gets a job offer, the generated email contains text fragments from typical job offer emails.
With this text generation mechanism, all texts in the event log are unique, but the words and sentences in the texts correlate with the path of the corresponding case.

\textbf{werk.nl} (Real-world) This event log describes customer's journeys of the Employee Insurance Agency commissioned by the Dutch Ministry of Social Affairs and Employment. The log is aggregated from two anonymized data sets, that were provided in the BPI Challenge 2016 \cite{bpichallenge2016}, containing click data of logged in customers from the official website werk.nl and phone call data from the call center.
Both data sets are joined based on the customer ID to derived a detailed view on customer contacts.
For each phone call the costumer's question is available as a text attribute in English. In addition, the customer's age (grouped) and gender is considered as additional attributes.
The event log is filtered to remove outlier activities (threshold <0.5\%) and infrequent trace variants (2 or less traces with the same variant). After prepossessing, 18 distinct activities and 1001 trace variants remain, such that the process can be described as quite unstructured.

 
\textbf{Help Desk} (Real-world)

\begin{table}[!htbp]
	\begin{tabularx}{\textwidth}{l r r r}
		\toprule
		\textbf{Data Set} & \textbf{Job Application} & \textbf{werk.nl} &\textbf{Help Desk}  \\
		\midrule
		Number of cases & & 15\,001& \\
		Number of trace variants & & 1\,001 & \\
		Number of events & & 55\,220 & \\
		Events per case (mean) & & 3.681& \\
		Median case duration (days) & & 0.224& \\
		Mean case duration (days)&&  0.713 & \\
		Number of activities & & 18 & \\
		Number of words (pre filtering) & &247\,010 & \\
		Number of words (post filtering)  & &98\,915 & \\
		Vocabulary size (pre filtering) & & 1\,203 & \\
		Vocabulary size (post filtering) & & 815 & \\
		\bottomrule
	\end{tabularx}
	\caption[Overview of evaluated data sets]{Overview of evaluated data sets.}
	\label{tab:logs}
\end{table}

\section{Evaluation Method}

Each event log is evaluated in a consistent procedure.
In the first step, the event log is separated into a training and test log. 
The training log consists of the first 2/3 chronologically ordered traces and is used to fit the prediction model to the historical data.
The remaining 1/3 of traces are used to measure the prediction performance.
For each trace $\sigma$ in the test log, all prefixes $hd^k(\sigma)$ of length $2 \leq k \leq |\sigma| - 1$ are considered as instances for prediction.
For prefix traces with only one event, predictions seem to be less stable as also observed in \cite{DBLP:conf/caise/TaxVRD17} and are therefore not included for the metric computation.
The LSTM network is trained with at most 100 epochs and the learning rate is initialized with 0.001.
During the training of the LSTM model, 20\% of the training log is used for validation: If the error on the validation log is not decreasing anymore for 5 epochs, the training rate is reduced, and if the error is not decreased for 10 epochs, the training is stopped in order to avoid overfitting.
Furthermore, the LSTM layers use dropout \cite{DBLP:journals/corr/abs-1207-0580} of 20\% during training as an additional measure against overfitting.

For classification (i.e. categorical prediction) task, like next event and outcome prediction, the accuracy is utilized as metric.
The accuracy is computed as the number of correct predictions $t$ divided by the total number of predictions $n$, i.e. 
\begin{equation*}
	\textrm{accuracy} = \dfrac{t}{n} = \dfrac{\textrm{\# correct predictions}}{\textrm{\# total predictions}}.
\end{equation*}

For regression tasks, like the next event time and the case cycle time predictions, the mean absolute error (MAE) is computed to measure the prediction performance. The mean absolute error indicates the average absolute difference between the predicted value $\hat{y}$ and the true value $y$,  precisely
\begin{equation*}
	\textrm{MAE} = \dfrac{1}{n}\sum_{i=1}^{n}|\hat{y_i} - y_i|.
\end{equation*}
This error metric is favored, since it gives a more intuitive interpretation and is less sensitive to outliers compared to similar metrics like the mean squared error (MSE).

The prediction model is evaluated with each presented text model and different encoding lengths for the text.
For the Bag of Words and Bag of N-Gram model the encoding length is adjusted by only considering the most frequent terms in the log.
The encoding length of the non-textual data depends on the considered attributes and their number of the distinct values in the event logs.
The Bag of N-Gram model is used with bigrams ($n=2$).
The Paragraph Vector model is trained for 15 epochs.

\section{Next Event Prediction}

\begin{table}[!htbp]
	\setlength\tabcolsep{3pt}
	\begin{tabularx}{\textwidth}{
			>{\hsize=1.0\hsize}X
			>{\hsize=1.0\hsize}X
			>{\hsize=1.0\hsize}X
			>{\hsize=1.0\hsize}X
			>{\hsize=1.0\hsize}X
			>{\hsize=1.0\hsize}X
			>{\hsize=1.0\hsize}X
			>{\hsize=1.0\hsize}X
		}
		\toprule
		 & & \multicolumn{2}{l}{\textbf{werk.nl}} & \multicolumn{2}{l}{\textbf{Job Application}} & \multicolumn{2}{l}{\textbf{Help Desk}} \\
		 Text \newline Model & Encoding Length &Accuracy & MAE & Accuracy& MAE  & Accuracy& MAE  \\
		\midrule
		\multicolumn{8}{c}{\textit{Text-Aware Process Prediction (LSTM)}} \\
		BoW& 50 &  && & \\
		BoW& 100 && &&& \\
		BoW& 500 &  &&& \\
		BoNG& 50 &  &&& \\
		BoNG& 100 &  &&& \\
		BoNG& 500 &  &&& \\
		PV& 10 & & && \\
		PV& 20 & & && \\
		PV& 100 & & && \\
		LDA& 10 & & & & \\
		LDA& 20 & & & &\\
		LDA& 100 & & && \\
		\multicolumn{8}{c}{\textit{LSTM baseline}} \\
		- & 0 & & && \\
		\bottomrule
	\end{tabularx}
	\caption[Experimental results for the next event prediction]{Experimental results for the next event prediction.}
	\label{tab:next-event}
\end{table}



\pgfplotscreateplotcyclelist{colorlist}{%
	cyan!60!black,every mark/.append style={fill=cyan!60!black},mark=*\\%1
	orange!60!black,every mark/.append style={fill=orange!60!black},mark=square*\\%2
	darkgray!60!black,every mark/.append style={fill=darkgray!60!black},mark=otimes*\\%3
	red!60!black,every mark/.append style={fill=red!60!black},mark=triangle*\\%4
	olive!60!black,every mark/.append style={fill=olive!60!black},mark=diamond*\\%5
}

\begin{figure}
	\centering
	\begin{subfigure}{.5\textwidth}
		\centering
		\begin{tikzpicture}[scale=0.9]
			\begin{axis}[
				title=Next Activity Prediction,
				xlabel={Prefix length},
				ylabel={Accuracy},
				legend pos=north west,
				cycle list name=colorlist,
				]
			\end{axis}
		\end{tikzpicture}
		\caption{A subfigure}
		\label{fig:sub1}
	\end{subfigure}%
	\begin{subfigure}{.5\textwidth}
		\centering
		\begin{tikzpicture}[scale=0.9]
			\begin{axis}[
				title=Next Event Time Prediction,
				xlabel={Prefix length},
				ylabel={Mean Absolute Error},
				legend pos=north west,
				cycle list name=colorlist,
				]
				\addplot table[x=x,y=a, col sep= comma] {data/example.csv};
				\addlegendentry{$a$}
				\addplot table[x=x,y=b, col sep= comma] {data/example.csv};
				\addlegendentry{$b$}
				\addplot table[x=x,y=c, col sep= comma] {data/example.csv};
				\addlegendentry{$c$}
				\addplot table[x=x,y=d, col sep= comma] {data/example.csv};
				\addlegendentry{$d$}
				\addplot table[x=x,y=e, col sep= comma] {data/example.csv};
				\addlegendentry{$e$}
			\end{axis}
		\end{tikzpicture}
		\caption{A subfigure}
		\label{fig:sub2}
	\end{subfigure}
	\caption[Next activity prediction] {Next activity prediction.}
	\label{fig:test}
\end{figure}



\section{Outcome and Case Cycle Time Prediction}

\begin{table}[!htbp]
	\begin{tabularx}{\textwidth}{l l l l }
		\toprule
		& & &  \\
		\midrule
		& & & \\
		\bottomrule
	\end{tabularx}
	\caption[Experimental results for the case cycle time prediction]{Experimental results for the case cycle time prediction.}
	\label{tab:outcome-cycle-time}
\end{table}


