
\section{Motivation}

The rapid growth of data generated by large-scale information systems leads to new opportunities for  society and businesses.
In order to turn the massive amount of data into value, efficient techniques are needed, that are able to extract useful information.
The generated value comes in different forms, for example visualizations, models, aggregated data or predictions can be of interest. 

A remarkable subset of this data is described as \textit{event data}, which is generated by \textit{process-aware information systems} in order to manage, execute and monitor business processes \cite{DBLP:journals/topnoc/Aalst09}.
With the non stopping rise of digitization of business processes more and more event data becomes utilizable, thus the potential value of this data is exploding.

The scientific engagement aiming to discover, analyze and improve real processes using event data led to \textit{process mining}. Process mining bridges the gab between the data-driven characteristic of data science and the process-centric view of process science \cite{DBLP:books/sp/Aalst16}.
The ongoing success of progress mining in research has been transferred to businesses, that successfully offer or utilize this technology.
Celonis, which is often considered as one of the biggest commercial providers of process mining, has been valued 2.5 billion dollar only 9 years after the company was founded \cite{celonis}.

Modern process mining software tends to focus on continuous analysis rather than the traditional offline and project-based approaches.
These business process monitoring systems are a key success factor for many organizations, since they allow to understand and supervise all connected processes of a company in real-time as the data is flowing.
The core idea of this approach is to automate process mining and keep a persistent connecting between the business process data and the analytical capabilities.

However, traditional process mining tends to be backward-looking  \cite{DBLP:conf/scsc/Aalst18}, i.e. it rather focuses on answering the question "What did happen?", rather than "What will happen?" or even "What should be done?".
Therefore, new techniques are required to add the forward-perspective to process mining software.

\section{Problem Statement}

Businesses can develop a competitive advantage, if their process mining software has predictive capabilities, that allow to predict the future of a running process instance.
For example, if it is known beforehand that a running process instance will probably exceed
its deadline, measures can be initiated before damage occurs.

Precisely, given an event log with past executions of a process and a running process instance, we would like to answer the following questions:

\begin{itemize}
	\item What will happen next? $\rightarrow$ Activity of next event
	\item When will it happen? $\rightarrow$ Timestamp of next event
	\item What is the outcome of the instance? $\rightarrow$ Activity of last event
	\item When will the instance finish $\rightarrow$ Timestamp of last event
	\item What is the most likely path of the instance $\rightarrow$  Activities and timestamps of all future events for the path that is most likely
\end{itemize}



\section{Research Goals}

This thesis aims to improve current state-of-art approaches for process prediction.
The main research goal is to design and implement a predictive model for event data that is able to take advantage of additional attribute and textual data associated with each event. 

\section{Contribution}



\section{Thesis Structure}

