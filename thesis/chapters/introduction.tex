
\section{Motivation}

The rapid growth of data generated by large-scale information systems leads to new opportunities for  society and businesses.
In order to turn the massive amount of data into value, efficient techniques are needed, that are able to extract new information.
The generated value can come in different forms: Depending on the context, visualizations, models, aggregated data or predictions can be of interest. 

A remarkable subset of this data is described as \textit{event data}, which is generated by software systems, that execute processes and save the executed activities in logs.
With the non stopping rise of digitization and processes switching from the analog world to the digital one, the theoretical value of this data is exploding.

This motivates scientist and business to take advantage of this potential.
The scientific engagement in this discipline is described by \textit{process mining}, which bridges the gab between the data-driven characteristic of data science and the process centric view of process science \cite{DBLP:books/sp/Aalst16}.
The ongoing success of progress mining in research has been transferred to business that successfully offer or utilize this technology.
Celonis, which can be considered as the biggest company in this field, has been valued 2.5 billion dollar only 9 years after the company was founded \cite{celonis}.



\section{Problem Statement}

Business process monitoring software is a key success factor for many businesses, since it allows to understand and supervise all connected processes of a company in real-time as the data is flowing in.
The core idea of this approach is to automate process mining and connect the business data directly with the analytical tool set.

However, traditional process mining tends to be backward-looking  \cite{DBLP:conf/scsc/Aalst18}, i.e. the focus is on answering the question 'What did happen?', rather than 'What will happen?' or even 'What should be done?'.

Businesses can develop a competitive advantage, if their process mining software has predictive capabilities, that allow to predict the future of a running process instance.
For example, if it is known beforehand that a running process instance will probably exceed
its deadline, measures can be initiated before damage occurs.

Precisely, given an event log with past executions of a process and a running process instance, we would like to answer the following questions:

\begin{itemize}
	\item What will happen next? $\rightarrow$ Activity of next event
	\item When will it happen? $\rightarrow$ Timestamp of next event
	\item What is the outcome of the instance? $\rightarrow$ Activity of last event
	\item When will the instance finish $\rightarrow$ Timestamp of last event
	\item What is the most likely path of the instance $\rightarrow$  Activities and timestamps of all future events for the path that is most likely
\end{itemize}



\section{Research Goals}

This thesis aims to improve current state-of-art approaches for process prediction.
The main research goal is to design and implement a predictive model for event data that is able to take advantage of additional attribute and textual data associated with each event. 

\section{Contribution}



\section{Thesis Structure}

